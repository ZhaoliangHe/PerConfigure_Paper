\section{Introduction}
\label{Section: introduction}
With the increasing demand for continuous video analysis in public safety and transportation, more and more cameras are being deployed to various locations. Video analysis can be completed according to the video analysis application built by different models, which can free the staff from complex and boring tasks or search through massive amounts of video data to find what you're looking for quickly. In recent years, we have also witnessed the emergence of a large number of excellent models for target detection.

For the collected video, the classical computer vision and deep neural network technology are generally used for video analysis. Video analysis applications consist of several video processing modules, typically including a decoder, a selective sampling frame application, and a target detector. Such an application always having multiple "knobs", such as frame rate, resolution, and model (for example, MobileNet, ResNet, or InceptionResNet), for different combinations of values of knobs is called different configurations. Since video analysis is a very complex process, we pay much attention to the consumption of resources in the calculation process, and the accuracy of inference is also our focus. Therefore, the problem that follows is how to balance the consumption of resources and accuracy. Choosing different configurations will affect the resource consumption and accuracy caused by video analysis. For example, using a complex model and high resolution can obviously accurately detect the target object, but it also requires more computing resources. However, choosing a simple model and low resolution can significantly reduce resource consumption, although it reduces the accuracy to some extent. And in the case of a highway video analysis, due to the rate of car travel cannot be predicted in advance, so when the car driving slowly(or static) because of the traffic jam, we can choose a lower frame rate (such as 1 FPS) without having to use a fixed on the whole video higher frame rate, this can significantly reduce resource consumption, but does not affect the accuracy of the video analysis. The best configuration for a video analytics pipeline alsovaries over time, often at a timescale of minutes or evenseconds\cite{jiang2018chameleon}, so our goal is to find a range of "best fit" configurations that takes up the minimum amount of computing resources and is accurate to the desired threshold. The most intuitive way to solve this problem is to find the best solution by exhaustive all configurations, but the number of possible configurations grows exponentially, and thousands of configurations can be combined with just a few knobs, so exhaustive configuration is a highly unrealistic approach.

In this paper, we propose an approach based on reinforcement learning, which can skillfully use the temporal and spatial characteristics of video to select the "most appropriate" configuration, thus solving a difficult optimal configuration decision problem (reduced to linear time complexity) in a very low-cost way:

\begin{itemize}
\item Time related
\item Spatial correlation
\item Independent configuration
\end{itemize}