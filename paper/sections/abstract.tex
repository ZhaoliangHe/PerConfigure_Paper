\begin{abstract}
 
Deep convolutional neural networks (NN)-based video analytics demands intensive computation resources and high inference accuracy.
A NN-based video analytics \emph{pipeline} includes multiple \emph{knobs}, such as frame rate, resolution and target detector model. A combination of the knob values is a video analytics \emph{configuration}. Due to the highly variable video content, the \emph{best} configuration for a video analytics pipeline also varies over time. Periodically profiling the processing pipeline to find an optimal resource-accuracy \emph{tradeoff} by exhaustive all configurations, it would be prohibitively expensive since the number of possible configurations is exponential in the number of knobs and their values. Searching a large space of configurations periodically causes an overwhelming resource overhead that far outstrips the gains of periodically profiling. Knowing this, designing an \emph{automatic} approach to decide what is the best configuration for the current video content is meaningful. 
In this paper, we propose a reinforcement learning (RL)-based automatic video analytics configuration framework, AutoConfigure. In particular, we design a reinforcement learning agent that learns the \emph{optimal} configuration based on its own experience to decide the best configuration for current video content. The unique feature of AutoConfigure is \emph{content-driven}, meaning that AutoConfigure can adapt the best configuration to intrusive dynamics of video contents. 
We implement and evaluate this approach in object detection task comparing its performance to static configuration and dynamic configuration baseline. We show that AutoConfigure achieves 20-30\%\textcolor{note}{(goal)} higher accuracy with the same amount of resources, or achieve the same accuracy with only 30-50\%\textcolor{note}{(goal)} of the resources. Furthermore, AutoConfigure proves to be more efficient than existing baselines by creating an overhead of less than 1\%\textcolor{note}{(goal)} to the overall video analytics pipeline.

\end{abstract}
